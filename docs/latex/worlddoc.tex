%% Generated by Sphinx.
\def\sphinxdocclass{report}
\documentclass[letterpaper,11pt,english,openany,oneside]{sphinxmanual}
\ifdefined\pdfpxdimen
   \let\sphinxpxdimen\pdfpxdimen\else\newdimen\sphinxpxdimen
\fi \sphinxpxdimen=.75bp\relax
\ifdefined\pdfimageresolution
    \pdfimageresolution= \numexpr \dimexpr1in\relax/\sphinxpxdimen\relax
\fi
%% let collapsible pdf bookmarks panel have high depth per default
\PassOptionsToPackage{bookmarksdepth=5}{hyperref}

\PassOptionsToPackage{warn}{textcomp}
\usepackage[utf8]{inputenc}
\ifdefined\DeclareUnicodeCharacter
% support both utf8 and utf8x syntaxes
  \ifdefined\DeclareUnicodeCharacterAsOptional
    \def\sphinxDUC#1{\DeclareUnicodeCharacter{"#1}}
  \else
    \let\sphinxDUC\DeclareUnicodeCharacter
  \fi
  \sphinxDUC{00A0}{\nobreakspace}
  \sphinxDUC{2500}{\sphinxunichar{2500}}
  \sphinxDUC{2502}{\sphinxunichar{2502}}
  \sphinxDUC{2514}{\sphinxunichar{2514}}
  \sphinxDUC{251C}{\sphinxunichar{251C}}
  \sphinxDUC{2572}{\textbackslash}
\fi
\usepackage{cmap}
\usepackage[T1]{fontenc}
\usepackage{amsmath,amssymb,amstext}
\usepackage{babel}



\usepackage{tgtermes}
\usepackage{tgheros}
\renewcommand{\ttdefault}{txtt}



\usepackage[Bjarne]{fncychap}
\usepackage{sphinx}

\fvset{fontsize=auto}
\usepackage{geometry}


% Include hyperref last.
\usepackage{hyperref}
% Fix anchor placement for figures with captions.
\usepackage{hypcap}% it must be loaded after hyperref.
% Set up styles of URL: it should be placed after hyperref.
\urlstyle{same}

\addto\captionsenglish{\renewcommand{\contentsname}{Interactables}}

\usepackage{sphinxmessages}
\setcounter{tocdepth}{4}
\setcounter{secnumdepth}{4}

        \usepackage{charter}
        \usepackage[defaultsans]{lato}
        \usepackage{inconsolata}
    

\title{WorldDoc}
\date{Apr 26, 2022}
\release{0.1}
\author{TwoBitMachines}
\newcommand{\sphinxlogo}{\vbox{}}
\renewcommand{\releasename}{Release}
\makeindex
\begin{document}

\pagestyle{empty}
\sphinxmaketitle
\pagestyle{plain}
\sphinxtableofcontents
\pagestyle{normal}
\phantomsection\label{\detokenize{index::doc}}


\sphinxstepscope


\chapter{Bridge}
\label{\detokenize{interactables/bridge:bridge}}\label{\detokenize{interactables/bridge::doc}}
\sphinxAtStartPar
A bridge creates dynamic movement. Characters can walk and jump on them.

\noindent{\hspace*{\fill}\sphinxincludegraphics[width=1.000\linewidth]{{BridgeGif}.png}\hspace*{\fill}}

\begin{sphinxadmonition}{tip}{Tip:}
\sphinxAtStartPar
Characters, by default, are enabled to interact with bridges. If this property
is not desired, disable it in the character’s collision settings to save
unnecessary collision checks.
\end{sphinxadmonition}


\begin{savenotes}\sphinxattablestart
\centering
\begin{tabular}[t]{|\X{25}{125}|\X{100}{125}|}
\hline
\sphinxstyletheadfamily 
\sphinxAtStartPar
Property
&\sphinxstyletheadfamily \\
\hline
\sphinxAtStartPar
Planks
&
\sphinxAtStartPar
The number of planks in the bridge.
\\
\hline
\sphinxAtStartPar
Gravity
&
\sphinxAtStartPar
The force of gravity acting on the bridge.
\\
\hline
\sphinxAtStartPar
Bounce
&
\sphinxAtStartPar
The force exerted on the bridge when interacting with characters.
\\
\hline
\sphinxAtStartPar
Stiffness
&
\sphinxAtStartPar
The larger the number, the less sag the bridge will have. For performance, keep this value below 20.
\\
\hline
\sphinxAtStartPar
Plank
&
\sphinxAtStartPar
Create a gameobject and add the plank’s sprite. Make this a child of the bridge gameobject and set the reference.
This will be used as a template to instantiate the remaining planks. Change the transform’s scale to achieve
the desired plank width. Lastly, the offset will shift each plank visually.
\\
\hline
\sphinxAtStartPar
Area
&
\sphinxAtStartPar
The system will check for plank collisions once the character is inside the bridge area.
The area width is set automatically, but the height must be specified. The offset will offset the area in the y direction.
\\
\hline
\sphinxAtStartPar
Create
&
\sphinxAtStartPar
Once all the settings are chosen, press this button to create the bridge. Anytime you change the bridge’s position or a setting, recreate the bridge to enact the changes.
\\
\hline
\sphinxAtStartPar
View
&
\sphinxAtStartPar
If enabled, the bridge gizmos will be visible.
\\
\hline
\end{tabular}
\par
\sphinxattableend\end{savenotes}

\begin{sphinxadmonition}{important}{Important:}
\sphinxAtStartPar
The start of the bridge corresponds to the transform’s position. Make sure the transform’s handle position is set to Pivot (and not to Center) for proper placement.
A scene handle tool, a red circle, is used to specify the end of the bridge. The distance between the start and end points determines the length of the bridge.
\end{sphinxadmonition}

\sphinxstepscope


\chapter{Rope}
\label{\detokenize{interactables/rope:rope}}\label{\detokenize{interactables/rope::doc}}
\sphinxAtStartPar
The player can use ropes to swing or for simple idle interactions.

\begin{sphinxadmonition}{note}{Note:}
\sphinxAtStartPar
The player’s Rope ability must be enabled to interact with ropes.
\end{sphinxadmonition}


\begin{savenotes}\sphinxattablestart
\centering
\begin{tabular}[t]{|\X{25}{125}|\X{100}{125}|}
\hline
\sphinxstyletheadfamily 
\sphinxAtStartPar
Property
&\sphinxstyletheadfamily \\
\hline
\sphinxAtStartPar
Type
&
\sphinxAtStartPar
If Swing is enabled, the player will swing on the rope. If Idle is enabled, the player
will pass through the rope, causing it to move.
\\
\hline
\sphinxAtStartPar
Rope End Radius
&
\sphinxAtStartPar
If Swing is enabled, once player and end tether are within this radius, the player will latch onto the rope automatically.
\\
\hline
\sphinxAtStartPar
Rope Radius
&
\sphinxAtStartPar
If Idle is enabled, the system will check for rope collisions if the player is inside this radius. The center of this radius is set automatically.
\\
\hline
\sphinxAtStartPar
Tether Radius
&
\sphinxAtStartPar
If Idle is enabled, it is the radius of each tether used to detect the player.
\\
\hline
\sphinxAtStartPar
Force
&
\sphinxAtStartPar
If Idle is enabled, it is the movement force applied to a tether upon interaction.
\\
\hline
\sphinxAtStartPar
Tethers
&
\sphinxAtStartPar
The number of tethers in the rope.
\\
\hline
\sphinxAtStartPar
Gravity
&
\sphinxAtStartPar
The force of gravity acting on the rope.
\\
\hline
\sphinxAtStartPar
Stiffness
&
\sphinxAtStartPar
The larger the number, the less sag the rope will have. For performance, keep this value below 20.
\\
\hline
\sphinxAtStartPar
Double Anchor
&
\sphinxAtStartPar
Both the start and end of the rope are anchored.
\\
\hline
\sphinxAtStartPar
Rope Start
&
\sphinxAtStartPar
Create a gameobject and add the tether’s sprite. Make this a child of the rope gameobject and set the reference.
This will be used as a template to instantiate the remaining tethers. Change the transform’s scale to achieve
the desired tether height;
\\
\hline
\sphinxAtStartPar
Rope End
&
\sphinxAtStartPar
Every time the rope is created, it destroys and recreates all the tethers. Sometimes the end tether contains components like Health. To prevent having to
add these components every time the rope is recreated, specify the end tether gameobject to prevent it from being destroyed.
\\
\hline
\sphinxAtStartPar
Create
&
\sphinxAtStartPar
Once all the settings are chosen, press this button to create the rope. Anytime you change the rope’s position or a setting, recreate the rope to enact the changes.
\\
\hline
\sphinxAtStartPar
View
&
\sphinxAtStartPar
If enabled, the rope gizmos will be visible.
\\
\hline
\end{tabular}
\par
\sphinxattableend\end{savenotes}

\begin{sphinxadmonition}{important}{Important:}
\sphinxAtStartPar
The start of the rope corresponds to the transform’s position. Make sure the transform’s handle position is set to Pivot (and not to Center) for proper placement.
A scene handle tool, a red circle, is used to specify the end of the rope. The distance between the start and end points determines the length of the rope.
\end{sphinxadmonition}


\begin{savenotes}\sphinxattablestart
\centering
\begin{tabular}[t]{|\X{45}{145}|\X{100}{145}|}
\hline
\sphinxstyletheadfamily 
\sphinxAtStartPar
Method
&\sphinxstyletheadfamily \\
\hline
\sphinxAtStartPar
ApplyImpactAtEnd (float directionX, float impact)
&
\sphinxAtStartPar
This will apply an impact force in the x direction to the end of the rope. This is automatically used by the player for swinging.
\\
\hline
\sphinxAtStartPar
ApplyImpact (float value, Vector2 direction)
&
\sphinxAtStartPar
Each tether contains the component Tether. This class contains this method. Call it to apply a force to a tether in the specified direction. Ignore the value parameter and instead set
the impact force in the inspector field of the Tether class.
\\
\hline
\sphinxAtStartPar
UnlatchEndAnchor ( )
&
\sphinxAtStartPar
If double anchor is set true, you can set it false by calling this method. The end anchor will become free, letting the rope fall down.
\\
\hline
\end{tabular}
\par
\sphinxattableend\end{savenotes}

\begin{sphinxadmonition}{tip}{Tip:}
\sphinxAtStartPar
It’s possible to add a Health and Collider component to each tether for further interaction. This can be useful if the rope needs to collide with Projectiles. The Health component
is equipped to call the ApplyImpact() and UnlatchEndAnchor() methods through Unity Events.
\end{sphinxadmonition}

\sphinxstepscope


\chapter{Water}
\label{\detokenize{interactables/water:water}}\label{\detokenize{interactables/water::doc}}
\sphinxAtStartPar
Water is a dynamic area where the player can float and swim.

\begin{DUlineblock}{0em}
\item[] 
\end{DUlineblock}

\begin{sphinxadmonition}{note}{Note:}
\sphinxAtStartPar
The player’s Swim ability must be enabled to interact with water.
\end{sphinxadmonition}


\begin{savenotes}\sphinxattablestart
\centering
\begin{tabular}[t]{|\X{25}{125}|\X{100}{125}|}
\hline
\sphinxstyletheadfamily 
\sphinxAtStartPar
Property
&\sphinxstyletheadfamily \\
\hline
\sphinxAtStartPar
Shape
&
\sphinxAtStartPar
If Square is enabled, the system renders the water using square blocks. No textures or sprites are required. If Round is enabled, the system renders the water using a Mesh Renderer, creating a more curved wave.
A Texture2D and Material are required.
\\
\hline
\sphinxAtStartPar
Type
&
\sphinxAtStartPar
If Float is enabled, the player will stay above the water line.
If Swim is enabled, the player can swim inside the water.
\\
\hline
\sphinxAtStartPar
Segments
&
\sphinxAtStartPar
The number of elements that create the water. The higher the number, the less blocky the water will look.
\\
\hline
\sphinxAtStartPar
Texture2D
&
\sphinxAtStartPar
If Shape mode is Round, provide the Texture2D that will be used to render the water.
\\
\hline
\sphinxAtStartPar
Material
&
\sphinxAtStartPar
If Shape mode is Round, provide the Material that will be used by the Mesh Renderer.
\\
\hline
\sphinxAtStartPar
Amplitude
&
\sphinxAtStartPar
The maximum height of the wave.
\\
\hline
\sphinxAtStartPar
Frequency
&
\sphinxAtStartPar
Dictates the number of waves in the water.
\\
\hline
\sphinxAtStartPar
Speed
&
\sphinxAtStartPar
How quickly a wave moves across the water.
\\
\hline
\sphinxAtStartPar
Spring
&
\sphinxAtStartPar
The force exerted on the water when interacting with the player.
\\
\hline
\sphinxAtStartPar
Damping
&
\sphinxAtStartPar
How quickly the spring force dissipates.
\\
\hline
\sphinxAtStartPar
Turbulence
&
\sphinxAtStartPar
This adds random noise into the water, creating a chaotic effect.
\\
\hline
\sphinxAtStartPar
Random Current
&
\sphinxAtStartPar
This will change the direction of the speed at intervals specified by this value. This value is randomized slightly to add unpredictability.
\\
\hline
\sphinxAtStartPar
Create
&
\sphinxAtStartPar
Once all the settings are chosen, press this button to create the body of water. Anytime you change the water’s position or a setting, recreate the water to enact the changes.
\\
\hline
\end{tabular}
\par
\sphinxattableend\end{savenotes}


\begin{savenotes}\sphinxattablestart
\centering
\begin{tabular}[t]{|\X{25}{125}|\X{100}{125}|}
\hline
\sphinxstyletheadfamily 
\sphinxAtStartPar
Body
&\sphinxstyletheadfamily 
\sphinxAtStartPar
For Square mode.
\\
\hline
\sphinxAtStartPar
Top
&
\sphinxAtStartPar
The color of the water line.
\\
\hline
\sphinxAtStartPar
Thickness
&
\sphinxAtStartPar
The thickness of the water line.
\\
\hline
\sphinxAtStartPar
Taper
&
\sphinxAtStartPar
The wave’s water line will be thicker at its crest, and thinner at its trough.
\\
\hline
\sphinxAtStartPar
Middle
&
\sphinxAtStartPar
The color at the middle of the water.
\\
\hline
\sphinxAtStartPar
Bottom
&
\sphinxAtStartPar
The color at the bottom of the water.
\\
\hline
\sphinxAtStartPar
Phase
&
\sphinxAtStartPar
The bottom of the water has wave like motion as well. Specify the phase of this wave.
\\
\hline
\sphinxAtStartPar
Offset
&
\sphinxAtStartPar
Offset the position of the bottom wave.
\\
\hline
\sphinxAtStartPar
Speed
&
\sphinxAtStartPar
How quickly the bottom wave moves across the water.
\\
\hline
\end{tabular}
\par
\sphinxattableend\end{savenotes}

\sphinxstepscope


\chapter{Ladder}
\label{\detokenize{interactables/ladder:ladder}}\label{\detokenize{interactables/ladder::doc}}
\sphinxAtStartPar
The humble ladder is used for climbing.

\begin{sphinxadmonition}{note}{Note:}
\sphinxAtStartPar
The player’s Ladder ability must be enabled to interact with ladders.
\end{sphinxadmonition}


\begin{savenotes}\sphinxattablestart
\centering
\begin{tabular}[t]{|\X{25}{125}|\X{100}{125}|}
\hline
\sphinxstyletheadfamily 
\sphinxAtStartPar
Property
&\sphinxstyletheadfamily \\
\hline
\sphinxAtStartPar
Size
&
\sphinxAtStartPar
The width and height of the ladder.
\\
\hline
\end{tabular}
\par
\sphinxattableend\end{savenotes}

\sphinxstepscope


\chapter{High Jump}
\label{\detokenize{interactables/highJump:high-jump}}\label{\detokenize{interactables/highJump::doc}}
\sphinxAtStartPar
Launch a character into the air upon contact.


\begin{savenotes}\sphinxattablestart
\centering
\begin{tabular}[t]{|\X{25}{125}|\X{100}{125}|}
\hline
\sphinxstyletheadfamily 
\sphinxAtStartPar
Property
&\sphinxstyletheadfamily \\
\hline
\sphinxAtStartPar
Jump Force
&
\sphinxAtStartPar
The amount of force launching a character.
\\
\hline
\sphinxAtStartPar
Radius
&
\sphinxAtStartPar
The radius of the collision circle. Once collision is made, the character will be launched.
The offset will change the collision’s center position in the y direction.
\\
\hline
\end{tabular}
\par
\sphinxattableend\end{savenotes}

\sphinxstepscope


\chapter{Foliage}
\label{\detokenize{interactables/foliage:foliage}}\label{\detokenize{interactables/foliage::doc}}
\sphinxAtStartPar
Decorate an environment with foliage to make it come alive. The foliage can sway with the wind and respond
to character movements.


\begin{savenotes}\sphinxattablestart
\centering
\begin{tabular}[t]{|\X{25}{125}|\X{100}{125}|}
\hline
\sphinxstyletheadfamily 
\sphinxAtStartPar
Property
&\sphinxstyletheadfamily \\
\hline
\sphinxAtStartPar
Jiggle
&
\sphinxAtStartPar
The motion effect produced when interacting with characters. Smaller values produce softer motions.
\\
\hline
\sphinxAtStartPar
Damping
&
\sphinxAtStartPar
How quickly the jiggle effect dissipates.
\\
\hline
\sphinxAtStartPar
Uniformity
&
\sphinxAtStartPar
The tendency for foliage to sway in the same direction if the foliage has the same y position.
\\
\hline
\sphinxAtStartPar
Wind Strength
&
\sphinxAtStartPar
The force of the wind swaying the foliage.
\\
\hline
\sphinxAtStartPar
Wind Frequency
&
\sphinxAtStartPar
How quickly the wind changes direction.
\\
\hline
\sphinxAtStartPar
Create Texture
&
\sphinxAtStartPar
Press this button to add a new Texture2D, which represents the foliage. Each Texture2D must have the same size as the specified Vector2 field, or else the tool will not work.
\\
\hline
\end{tabular}
\par
\sphinxattableend\end{savenotes}

\begin{sphinxadmonition}{warning}{Warning:}
\sphinxAtStartPar
The system groups all the Texture2D images of the foliage into an array. Thus, every Texture2D must be of the same size and share the same settings for
this process to work correctly. As a reminder, this component is working with Texture2D and not Sprites.
\end{sphinxadmonition}


\begin{savenotes}\sphinxattablestart
\centering
\begin{tabular}[t]{|\X{25}{125}|\X{100}{125}|}
\hline
\sphinxstyletheadfamily 
\sphinxAtStartPar
Texture2D
&\sphinxstyletheadfamily \\
\hline
\sphinxAtStartPar
Texture2D
&
\sphinxAtStartPar
The current Texture2D image of the foliage. The delete button will remove this Texture2D and all of its instances from the scene.
\\
\hline
\sphinxAtStartPar
Orientation
&
\sphinxAtStartPar
This determines what vertices to sway. If Bottom is enabled, place foliage on ground. If Top is enabled, place foliage on a ceiling. If Left or Right are enabled, place foliage on walls.
\\
\hline
\sphinxAtStartPar
Depth
&
\sphinxAtStartPar
Specify the rendering order of the Texture2D images relative to each other. As of now, there is no way specify a sorting layer.
The player is either in front or in back of the foliage \textendash{} never in between.
\\
\hline
\sphinxAtStartPar
Interaction
&
\sphinxAtStartPar
Choose how active the foliage is with character interactions. Maybe some foliage are dense and don’t need to sway as much as others. A value of zero will disable all interactions with characters.
\\
\hline
\end{tabular}
\par
\sphinxattableend\end{savenotes}


\begin{savenotes}\sphinxattablestart
\centering
\begin{tabular}[t]{|\X{25}{125}|\X{100}{125}|}
\hline
\sphinxstyletheadfamily 
\sphinxAtStartPar
Paint Brushes
&\sphinxstyletheadfamily 
\sphinxAtStartPar
Place foliage in the scene with brushes.
\\
\hline
\sphinxAtStartPar
Single Brush
&
\sphinxAtStartPar
Place a single foliage image.
\\
\hline
\sphinxAtStartPar
Random Brush
&
\sphinxAtStartPar
Choose as many foliage images as desired and drag the brush in the scene. The density value specifies how many images the brush can place per position.
\\
\hline
\sphinxAtStartPar
Eraser
&
\sphinxAtStartPar
Use this brush to erase foliage images.
\\
\hline
\sphinxAtStartPar
Instances
&
\sphinxAtStartPar
Every Foliage component can only have a maximum of 1023 images in the scene.
\\
\hline
\end{tabular}
\par
\sphinxattableend\end{savenotes}

\begin{sphinxadmonition}{tip}{Tip:}
\sphinxAtStartPar
If the brush tool is active, right click in the scene or repress the current brush button to deactivate it.
\end{sphinxadmonition}

\begin{sphinxadmonition}{note}{Note:}
\sphinxAtStartPar
The foliage system was designed with performance in mind. All foliage instances exist in code only (they’re not gameobjects), and the character interactions
are handled by Unity’s Job System.
\end{sphinxadmonition}

\sphinxstepscope


\chapter{Jump}
\label{\detokenize{playerAbilities/jump:jump}}\label{\detokenize{playerAbilities/jump::doc}}
\sphinxAtStartPar
The most fundamental ability in any platformer.

\begin{sphinxadmonition}{note}{Note:}
\sphinxAtStartPar
Jump height and jump time are set in the Collision settings in order to
calculate the force of gravity. However, this ability must still be enabled
if the player is required to jump.
\end{sphinxadmonition}


\begin{savenotes}\sphinxattablestart
\centering
\begin{tabular}[t]{|\X{25}{125}|\X{100}{125}|}
\hline
\sphinxstyletheadfamily 
\sphinxAtStartPar
Property
&\sphinxstyletheadfamily \\
\hline
\sphinxAtStartPar
Button Trigger
&
\sphinxAtStartPar
On user button interaction, choose exactly when the player jumps.
\\
\hline
\sphinxAtStartPar
Min Jump Height
&
\sphinxAtStartPar
If this value is greater than zero, the player will have a variable jump height, and min jump will be the lowest
jump height possible.
\\
\hline
\sphinxAtStartPar
Air Jumps
&
\sphinxAtStartPar
The number of extra jumps the player can perform in the air.
\\
\hline
\end{tabular}
\par
\sphinxattableend\end{savenotes}

\begin{sphinxadmonition}{important}{Important:}
\sphinxAtStartPar
If an ability already contains a jump force, do not add the Jump ability as an exception. For instance, the Wall ability
contains a few jumping options that it will execute internally. The Jump ability is geared for jumping on ground.
\end{sphinxadmonition}

\sphinxstepscope


\chapter{Dash}
\label{\detokenize{playerAbilities/dash:dash}}\label{\detokenize{playerAbilities/dash::doc}}
\sphinxAtStartPar
Increase the speed of the player to quickly cover distance.


\begin{savenotes}\sphinxattablestart
\centering
\begin{tabular}[t]{|\X{25}{125}|\X{100}{125}|}
\hline
\sphinxstyletheadfamily 
\sphinxAtStartPar
Property
&\sphinxstyletheadfamily \\
\hline
\sphinxAtStartPar
Buttons
&
\sphinxAtStartPar
The buttons that need to be tapped in order to trigger a dash.
\\
\hline
\sphinxAtStartPar
Dash Direction
&
\sphinxAtStartPar
If Horizontal Axis is enabled, the dash will occur along the x axis. In this state, only the left and right buttons are used. It is also possible
to use only one button and leave the other empty. If Multi Directional is enabled, all the buttons that are set will be utilized to move the player in one of
eight directions along the x and y axis.
\\
\hline
\sphinxAtStartPar
Button Taps
&
\sphinxAtStartPar
If Single Tap is enabled, pressing the button only once will trigger a dash. If Double Tap is enabled, pressing the button twice is required
to trigger a dash.
\\
\hline
\sphinxAtStartPar
Tap Threshold
&
\sphinxAtStartPar
If Double Tap is enabled, the threshold is the time interval in which the double tap must occur for the dash to trigger successfully.
\\
\hline
\sphinxAtStartPar
Duration
&
\sphinxAtStartPar
If Instant is enabled, the player will traverse the dash distance in one frame. If Incremental is enabled, the player will traverse the dash distance
according to the dash time.
\\
\hline
\sphinxAtStartPar
Dash Time
&
\sphinxAtStartPar
The time it will take to traverse the dash distance;
\\
\hline
\sphinxAtStartPar
Dash Distance
&
\sphinxAtStartPar
The total distance traversed while dashing.
\\
\hline
\sphinxAtStartPar
Cool Down
&
\sphinxAtStartPar
The time interval before the next dash can be triggered.
\\
\hline
\sphinxAtStartPar
On Ground Only
&
\sphinxAtStartPar
If enabled, the player must be on the ground in order to begin a dash.
\\
\hline
\sphinxAtStartPar
Cool Down
&
\sphinxAtStartPar
If enabled, the force of gravity will not affect a dash.
\\
\hline
\end{tabular}
\par
\sphinxattableend\end{savenotes}

\sphinxstepscope


\chapter{Hover}
\label{\detokenize{playerAbilities/hover:hover}}\label{\detokenize{playerAbilities/hover::doc}}
\sphinxAtStartPar
Escape gravity by letting the player hover in the air.


\begin{savenotes}\sphinxattablestart
\centering
\begin{tabular}[t]{|\X{25}{125}|\X{100}{125}|}
\hline
\sphinxstyletheadfamily 
\sphinxAtStartPar
Property
&\sphinxstyletheadfamily \\
\hline
\sphinxAtStartPar
Thrust
&
\sphinxAtStartPar
The forced used to propel the player upward. This force will be proportional to the jump force.
\\
\hline
\sphinxAtStartPar
Maintain
&
\sphinxAtStartPar
The tendency for the player to remain in the air. A value of one will prevent the player from descending downward, unless
the descend button is pressed.
\\
\hline
\sphinxAtStartPar
Thrust Button
&
\sphinxAtStartPar
Press this button to create thrust.
\\
\hline
\sphinxAtStartPar
Descend
&
\sphinxAtStartPar
The force that will drive the player downward. The descend button is optional. If it’s not used,
the player will descend on its own according to the maintain value.
\\
\hline
\sphinxAtStartPar
Descend Button
&
\sphinxAtStartPar
Press this button to create downward thrust.
\\
\hline
\sphinxAtStartPar
Exit
&
\sphinxAtStartPar
If On Ground Hit is enabled, the player will exit the hover state when the player touches the ground. If Button is enabled,
the player will exit the hover state when the specified button is pressed.
\\
\hline
\sphinxAtStartPar
Air Friction X
&
\sphinxAtStartPar
The air resistance applied to the player while hovering in the x direction.
\\
\hline
\sphinxAtStartPar
On Thrust
&
\sphinxAtStartPar
Every time the thrust button is pressed, this Unity Event will be invoked.
\\
\hline
\sphinxAtStartPar
On Descend
&
\sphinxAtStartPar
Every time the descend button is pressed, this Unity Event will be invoked.
\\
\hline
\end{tabular}
\par
\sphinxattableend\end{savenotes}

\sphinxstepscope


\chapter{Swim}
\label{\detokenize{playerAbilities/swim:swim}}\label{\detokenize{playerAbilities/swim::doc}}
\sphinxAtStartPar
Allow the player to swim or float on any body of water. The body of water will determine
if the player either floats or swims. If floating, the player will remain above the water line. If swimming, the player
will swim inside the body of water.

\begin{sphinxadmonition}{note}{Note:}
\sphinxAtStartPar
The player must have the Swim ability enabled to interact with water.
\end{sphinxadmonition}


\begin{savenotes}\sphinxattablestart
\centering
\begin{tabular}[t]{|\X{25}{125}|\X{100}{125}|}
\hline
\sphinxstyletheadfamily 
\sphinxAtStartPar
Property
&\sphinxstyletheadfamily \\
\hline
\sphinxAtStartPar
Spring
&
\sphinxAtStartPar
If floating, when the player enters the water, it will oscillate on the water line before coming to a rest.
This force dictates how quickly the oscillations occur.
\\
\hline
\sphinxAtStartPar
Damping
&
\sphinxAtStartPar
How quickly the spring force dissipates.
\\
\hline
\sphinxAtStartPar
Weight
&
\sphinxAtStartPar
How quickly the player sinks while swimming.
\\
\hline
\sphinxAtStartPar
Water Impact
&
\sphinxAtStartPar
The force exerted on the water upon entry. The force exerted while the player moves in the water will be proportional to this value and the player’s velocity.
\\
\hline
\sphinxAtStartPar
Water Friction X
&
\sphinxAtStartPar
Water resistance applied to the player in the x direction.
\\
\hline
\sphinxAtStartPar
Water Friction Y
&
\sphinxAtStartPar
Water resistance applied to the player in the y direction.
\\
\hline
\sphinxAtStartPar
Jump
&
\sphinxAtStartPar
The force used to jump out of the water.
\\
\hline
\sphinxAtStartPar
Switch Button
&
\sphinxAtStartPar
If water Switch Type is set to Yes, holding this button will transition the player from a floating state to a swimming state. To return to a floating state, the player
must reach the top of the water.
\\
\hline
\sphinxAtStartPar
On Enter Water
&
\sphinxAtStartPar
On water entry, a Unity Event containing the entry position is invoked. This could be useful for adding particle effects.
\\
\hline
\sphinxAtStartPar
On Exit Water
&
\sphinxAtStartPar
On water exit, a Unity Event containing the exit position is invoked.
\\
\hline
\end{tabular}
\par
\sphinxattableend\end{savenotes}

\sphinxstepscope


\chapter{Ladder}
\label{\detokenize{playerAbilities/ladderClimb:ladder}}\label{\detokenize{playerAbilities/ladderClimb::doc}}
\sphinxAtStartPar
The player can interact with ladders.


\begin{savenotes}\sphinxattablestart
\centering
\begin{tabular}[t]{|\X{25}{125}|\X{100}{125}|}
\hline
\sphinxstyletheadfamily 
\sphinxAtStartPar
Property
&\sphinxstyletheadfamily \\
\hline
\sphinxAtStartPar
Latch
&
\sphinxAtStartPar
If Automatic is enabled, the player will automatically latch to the ladder on contact, provided the player has a negative y velocity and a zero x velocity.
If Enter Button is enabled, specify the button that must be pressed in order for the player to latch onto the ladder.
\\
\hline
\sphinxAtStartPar
Climb
&
\sphinxAtStartPar
If Manual is enabled, specify the buttons (Up, Down) for climbing the ladder. If Automatic is enabled, the player will climb
the ladder automatically.
\\
\hline
\sphinxAtStartPar
Climb Speed
&
\sphinxAtStartPar
How quickly the player climbs the ladder.
\\
\hline
\sphinxAtStartPar
Stand On Top
&
\sphinxAtStartPar
If enabled, the player can stand on top of the ladder.
\\
\hline
\sphinxAtStartPar
Align To Center
&
\sphinxAtStartPar
If enabled, the player’s x position will align with the center of the ladder.
\\
\hline
\end{tabular}
\par
\sphinxattableend\end{savenotes}

\sphinxstepscope


\chapter{Rope}
\label{\detokenize{playerAbilities/ropeSwing:rope}}\label{\detokenize{playerAbilities/ropeSwing::doc}}
\sphinxAtStartPar
The player can interact with ropes.


\begin{savenotes}\sphinxattablestart
\centering
\begin{tabular}[t]{|\X{25}{125}|\X{100}{125}|}
\hline
\sphinxstyletheadfamily 
\sphinxAtStartPar
Property
&\sphinxstyletheadfamily \\
\hline
\sphinxAtStartPar
Swing Strength
&
\sphinxAtStartPar
The force added to the swing motion.
\\
\hline
\sphinxAtStartPar
Jump
&
\sphinxAtStartPar
If latched, the force used to jump away from the rope.
\\
\hline
\end{tabular}
\par
\sphinxattableend\end{savenotes}



\renewcommand{\indexname}{Index}
\printindex
\end{document}